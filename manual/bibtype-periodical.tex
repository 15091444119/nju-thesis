%%%%%%%%%%%%%%%%%%%%%%%%%%%%%%%%%%%%%%%%%%%%%%%%%%%%%%%%%%%%%%%%%%%%%%%%%%%%%%%
%%  
%% 文档类 NJU-Thesis 用户手册
%%
%% 作者:胡海星,starfish (at) gmail (dot) com
%% 项目主页: https://github.com/Haixing-Hu/nju-thesis
%%
%% This file may be distributed and/or modified under the conditions of the
%% LaTeX Project Public License, either version 1.2 of this license or (at your
%% option) any later version. The latest version of this license is in:
%%
%% http://www.latex-project.org/lppl.txt
%%
%% and version 1.2 or later is part of all distributions of LaTeX version
%% 1999/12/01 or later.
%%
%%%%%%%%%%%%%%%%%%%%%%%%%%%%%%%%%%%%%%%%%%%%%%%%%%%%%%%%%%%%%%%%%%%%%%%%%%%%%%%

\subsection{期刊}\label{subsec:bibtype-periodical}

期刊是指一种载有卷期号或年月顺序号、计划无限期地连续出版发行的出版物\cite{gbt7714-2005}。
它所对应的{\BibTeX}文献项类型为|periodical|;对应的\std{GB/T 3469-1983}文献类型单
字码为|J|\cite{gbt3469-1983}。

|periodical|不是{\BibTeX}的标准类型,而是{\njuthesis}的扩展。之所以用|periodical|这
个名称而不是|journal|,是因为|journal|已经是{\BibTeX}的标准字段名,为避免重名冲突,
所以只能叫|periodical|。

\begin{note}
如果引文是整个期刊的某一期或某几期,则应使用此类型; 如果引用的文献是期刊中的文
章,请使用|article|类型,具体参见\ref{subsec:bibtype-article}。
\end{note}

\subsubsection{必需字段}

\begin{itemize}
\item |author|:表示期刊的发行者,通常是一个机构。其格式参见\ref{subsec:bibfield-author}。
\item |title|:表示期刊的标题,其格式参见\ref{subsec:bibfield-title}。
\item |address|:表示期刊的出版地,其格式参见\ref{subsec:bibfield-address}。如果
  该字段不存在,{\BibTeX}排版时将会用``[S.l.]''或``[出版地不详]''替代。
\item |publisher|:表示期刊的出版者,其格式参见\ref{subsec:bibfield-publisher}。
  如果该字段不存在,{\BibTeX}排版时将会用``[s.n.]''或``[出版者不详]''替代。
\item |year|表示所引用的期刊的出版年,或所引用的一系列连续期刊的出版年的范围,其格式
  参见\ref{subsec:bibfield-year}。
\end{itemize}

\subsubsection{可选字段}

\begin{itemize}
\item |volume|: 表示所引用的期刊的卷号,或所引用的一系列连续期刊的卷号范围,其格式参
  见\ref{subsec:bibfield-volume}。
\item |number|: 表示所引用的期刊的期号,或所引用的一系列连续期刊的期号范围,其格式参
  见\ref{subsec:bibfield-number}。
\item |citedate|:表示期刊的在线版本的引用日期,其格式参见\ref{subsec:bibfield-citedate}。
\item |url|:表示期刊的在线版本的引用URL,其格式参见\ref{subsec:bibfield-url}。
\item |doi|:表示期刊的DOI编码,其格式参见\ref{subsec:bibfield-doi}
\item |language|:表示期刊的语言,其格式参见\ref{subsec:bibfield-language}。若语
  言为中文,此项必须填|zh|;否则,此项可省略。
\item 其他字段将不起作用。
\end{itemize}

