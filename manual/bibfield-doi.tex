%%%%%%%%%%%%%%%%%%%%%%%%%%%%%%%%%%%%%%%%%%%%%%%%%%%%%%%%%%%%%%%%%%%%%%%%%%%%%%%
%%  
%% 文档类 NJU-Thesis 用户手册
%%
%% 作者:胡海星,starfish (at) gmail (dot) com
%% 项目主页: https://github.com/Haixing-Hu/nju-thesis
%%
%% This file may be distributed and/or modified under the conditions of the
%% LaTeX Project Public License, either version 1.2 of this license or (at your
%% option) any later version. The latest version of this license is in:
%%
%% http://www.latex-project.org/lppl.txt
%%
%% and version 1.2 or later is part of all distributions of LaTeX version
%% 1999/12/01 or later.
%%
%%%%%%%%%%%%%%%%%%%%%%%%%%%%%%%%%%%%%%%%%%%%%%%%%%%%%%%%%%%%%%%%%%%%%%%%%%%%%%%

\subsection{doi}\label{subsec:bibfield-doi}

|doi|字段表示文献的``数字对象标识号(Digital Object Identifier)''。例如:
\begin{itemize}
\item |doi = {10.1007/s00223-003-0070-0}|
\end{itemize}

数字对象识别号是一套识别数字资源的机制,涵括的对象有视频、报告或书籍等等。它既有一套为资源命名的
机制,也有一套将识别号解析为具体地址的协议。

发展DOI的动机在于补充URI之不足,因为一方面URI指涉的URL经常变动,另一方面,URI表达的其实是资源
所在地(即网址),而非数字资源本身的信息。DOI能克服这两个问题。

一个DOI识别号经过解析后,可以连至一个或更多的数据。但识别号本身与解析后导向的数据并不相干,也可
能发生无法取得全部数据,只能得到相关出版品信息的情形。DOI的解析协议见诸RFC 3652,RFC 3651描
述命名机制,RFC 3650描述的则是其架构。DOI通过Handle系统解析识别号,但实际应用上大多是通过网
站解析.例如连进网址\url{http://dx.doi.org/10.1007/s00223-003-0070-0},就能看到对应识
别号\texttt{10.1007/s00223-003-0070-0}的论文信息或全文。

如果某文献项只有|doi|字段但没有|url|字段,该文献也被认为是在线文献。{\BibTeX}处理后,会在
其文献类型后增加``/OL'',表示其属于在线文献;同时也会自动根据DOI的解析网址前缀
\url{http://dx.doi.org/}合成其在线URL地址。

如果某文献项有|doi|字段,则它也应该有对应的|citedate|字段。

