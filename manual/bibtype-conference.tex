%%%%%%%%%%%%%%%%%%%%%%%%%%%%%%%%%%%%%%%%%%%%%%%%%%%%%%%%%%%%%%%%%%%%%%%%%%%%%%%
%%  
%% 文档类 NJU-Thesis 用户手册
%%
%% 作者:胡海星,starfish (at) gmail (dot) com
%% 项目主页: https://github.com/Haixing-Hu/nju-thesis
%%
%% This file may be distributed and/or modified under the conditions of the
%% LaTeX Project Public License, either version 1.2 of this license or (at your
%% option) any later version. The latest version of this license is in:
%%
%% http://www.latex-project.org/lppl.txt
%%
%% and version 1.2 or later is part of all distributions of LaTeX version
%% 1999/12/01 or later.
%%
%%%%%%%%%%%%%%%%%%%%%%%%%%%%%%%%%%%%%%%%%%%%%%%%%%%%%%%%%%%%%%%%%%%%%%%%%%%%%%%

\subsection{会议录}\label{subsec:bibtype-conference}

会议录是专著的一种,它是指在一定范围的学术会议和专业性会议后,将会上宣读、讨论和散发的论文或报告,
加以编辑出版的文献\cite{hudong2013huiyilu}。会议也包括座谈会、讨论会等\cite{gbt7714-2005}。
会议录所对应的{\BibTeX}文献项类型为|proceedings|或|conference|;所对应的\std{GB/T 3469-1983}
文献类型单字码为|C|\cite{gbt7714-2005}。

\begin{note}
中文中有时也将“会议录”称为“会议论文集”。注意“会议论文集”和一般意义上的“论文集”是不同的:
“会议论文集”就是指“会议录”,而“论文集”则是指关于某一主题的论文汇编,或同一个作者的论文
汇编。所以“会议录”或“会议论文集”应该使用|proceedings|或|conference|类型,对应的
\std{GB/T 3469-1983}文献类型单字码为|C|;而“论文集”应该使用|collection|类型,对应
的\std{GB/T 3469-1983}文献类型单字码为|G|。关于|collection|类型,
参见\ref{subsec:bibtype-collection}。
\end{note}

\subsubsection{必需字段}

\begin{itemize}
\item |editor|:表示该会议录的编辑者,其格式参见\ref{subsec:bibfield-editor};
\item |title|:表示该会议录的标题,其格式参见\ref{subsec:bibfield-title};
\item |year|:表示该会议录的出版年,其格式参见\ref{subsec:bibfield-year}。
\end{itemize}

\begin{note}
会议录的主要责任者应为其编辑者,应填写在|editor|字段中,而|author|字段将被忽略。
\end{note}

\subsubsection{可选字段}

\begin{itemize}
\item |translator|:表示该会议录的翻译者,其格式参见\ref{subsec:bibfield-translator};
\item |series|:表示该会议录所属的丛书的标题,其格式参见\ref{subsec:bibfield-series};
\item |volume|:表示该会议录在其所属的丛书或多卷书中的卷号,其格式参见\ref{subsec:bibfield-volume};
\item |edition|:表示该会议录的版本,其格式参见\ref{subsec:bibfield-edition};
\item |address|:表示该会议录的出版地,其格式参见\ref{subsec:bibfield-address};
\item |publisher|:表示该会议录的出版社,其格式参见\ref{subsec:bibfield-publisher};
\item |pages|:表示引文在该会议录中所处的页码或页码范围,其格式参见\ref{subsec:bibfield-pages};
\item |citedate|:表示该会议录的在线版本的引用日期,其格式参见\ref{subsec:bibfield-citedate};
\item |url|:表示该会议录的在线版本的引用URL,其格式参见\ref{subsec:bibfield-url};
\item |doi|:表示该会议录的DOI编码,其格式参见\ref{subsec:bibfield-doi};
\item |language|:表示该会议录的语言,其格式参见\ref{subsec:bibfield-language};
\item 其他字段将不起作用。
\end{itemize}

\begin{note}
会议录所属丛书的标题不应出现在|title|字段中,而应填写在|series|字段中。
会议录所属多卷书的卷号也不应出现在|title|字段中,而应填写在|volume|字段中。
若该会议录的语言为中文,则必须将|language|字段设置为|zh|;否则可忽略|language|字段。
\end{note}

\subsubsection{例子}

\begin{verbatim}
@proceedings{yufin2000,
  editor={Yufin, S. A.},
  title={Geoecology and computers: Proceedings of the Third 
         International Conference on Advance of Computer Methods 
         in Geotechnical and Geoenvironmental Engineering, 
         Moscow, Russia, February 1--4, 2000},
  address={Rotterdam},
  publisher={A. A. Balkema},
  year={2000},
}

@proceedings{rosenthall1963,
  editor={Rosenthall, E. M.},
  title={Proceedings of the Fifth Canadian Mathematical Congress,
         University of Montreal, 1961},
  address = {Toronto},
  publisher = {University of Toronto Press},
  year = {1963},
}

@conference{ganzha2000,
  editor = {Ganzha, V. G. and Mayr, E. W. and Vorozhtsov, E. V.},
  title = {Computer algebra in scientific computing: Proceedings
       of the Third Workshop on Algebra in Scientific Computing, 
       Samarkand, October 5-9,2000},
  address = {Berlin},
  publisher = {Springer},
  year = {c2000},
}

@proceedings{zhlixue1990,
  editor={{中国力学学会}},
  title={第3届全国实验流体力学学术会议论文集},
  address={天津},
  year={1990},
  language={zh},
}
\end{verbatim}



