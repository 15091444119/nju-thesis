\chapter{参考文献数据库}\label{chap:bib}

%%%%%%%%%%%%%%%%%%%%%%%%%%%%%%%%%%%%%%%%%%%%%%%%%%%%%%%%%%%%%%%%%%%%%%%%%%%%%%%
\section{简介}\label{sec:bib-intro}


%%%%%%%%%%%%%%%%%%%%%%%%%%%%%%%%%%%%%%%%%%%%%%%%%%%%%%%%%%%%%%%%%%%%%%%%%%%%%%%
\section{格式}\label{sec:bib-format}

%%%%%%%%%%%%%%%%%%%%%%%%%%%%%%%%%%%%%%%%%%%%%%%%%%%%%%%%%%%%%%%%%%%%%%%%%%%%%%%
\section{文献类型}\label{sec:bib-type}

\subsection{书籍}

书籍类文献所对应的{\BibTeX}文献项类型为|book|。

\subsubsection{必需字段}

\begin{itemize}
\item |author|:表示书籍的作者,其格式参见\ref{subsec:author}。此字段和|editor|字段至少
要有一个存在,也可以两个都存在。
\item |editor|:表示书籍的编辑者,其格式参见\ref{subsec:editor}。此字段和|author|字段至
少要有一个存在,也可以两个都存在。
\item |title|:表示书籍的标题,其格式参见\ref{subsec:title}。主标题和副标题之间用冒号
``:''隔开。
\item |address|:表示书籍的出版地,其格式参见\ref{subsec:address}。如果该字段不存在,
{\BibTeX}排版时将会用``[S.l.]''或``[出版地不详]''替代。
\item |publisher|:表示书籍的出版者,其格式参见\ref{subsec:publisher}。如果该字段不存在,
{\BibTeX}排版时将会用``[s.n.]''或``[出版者不详]''替代。
\item |year|表示书籍的出版年代,其格式参见\ref{subsec:year}。
\end{itemize}

\subsubsection{可选字段}

\begin{itemize}
\item |translator|:表示书籍的翻译者,其格式参见\ref{subsec:translator}。
\item |edition|:表示书籍的版本,其格式参见\ref{subsec:edition}。
\item |pages|表示引文所处的页码或页码范围,其格式参见\ref{subsec:pages}。
\item |citedate|:表示在线版书籍的引用日期,其格式参见\ref{subsec:citedate}。
\item |url|:表示在线版书籍的引用URL,其格式参见\ref{subsec:url}。
\item |doi|:表示书籍的DOI编码,其格式参见\ref{subsec:doi}
\item |language|:表示书籍的语言,其格式参见\ref{subsec:language}。若书籍为中文,此项
必须填|zh|;否则,此项可省略。
\item 其他字段将不起作用。
\end{itemize}

\subsection{书籍中的析出文献}

书籍中的析出文献所对应的{\BibTeX}文献项类型为|incollection|。

\subsubsection{必需字段}

\begin{itemize}
\item |author|:表示引文的作者,其格式参见\ref{subsec:author}。
\item |title|:表示引文的标题,其格式参见\ref{subsec:title}。
\item |booktitle|:表示引文所在书籍的标题,其格式参见\ref{subsec:title}。
\item |editor|:表示引文所在书籍的编辑者或作者,其格式参见\ref{subsec:editor}。
\item |address|:表示引文所在书籍的出版地,其格式参见\ref{subsec:address}。如果该字段不存在,
{\BibTeX}排版时将会用``[S.l.]''或``[出版地不详]''替代。
\item |publisher|:表示引文所在书籍的出版者,其格式参见\ref{subsec:publisher}。如果该字段不存在,
{\BibTeX}排版时将会用``[s.n.]''或``[出版者不详]''替代。
\item |year|表示引文所在书籍的出版年代,其格式参见\ref{subsec:year}。
\item |pages|表示引文所处的页码或页码范围,其格式参见\ref{subsec:pages}。
\end{itemize}

\subsubsection{可选字段}

\begin{itemize}
\item |translator|:表示书籍的翻译者,其格式参见\ref{subsec:translator}。
\item |edition|:表示书籍的版本,其格式参见\ref{subsec:edition}。
\item |citedate|:表示在线版书籍的引用日期,其格式参见\ref{subsec:citedate}。
\item |url|:表示在线版书籍的引用URL,其格式参见\ref{subsec:url}。
\item |doi|:表示书籍的DOI编码,其格式参见\ref{subsec:doi}
\item |language|:表示书籍的语言,其格式参见\ref{subsec:language}。若书籍为中文,此项
必须填|zh|;否则,此项可省略。
\item 其他字段将不起作用。
\end{itemize}

\subsection{小册子}

\subsection{手册}

\subsection{学位论文}


\subsection{}

\subsection{期刊论文}

\subsection{会议论文}

\subsection{技术报告}

\subsection{未发表论文}

\subsection{标准}

\subsection{专利}

\subsection{网页}

%%%%%%%%%%%%%%%%%%%%%%%%%%%%%%%%%%%%%%%%%%%%%%%%%%%%%%%%%%%%%%%%%%%%%%%%%%%%%%%

\section{文献项字段}\label{sec:bib-field}

\subsection{language}\label{subsec:language}

|language|字段用于说明该文献项对应的文献所使用的语言。例如:
\begin{itemize}
\item |language = {zh}|
\item |language = {en}|
\end{itemize}

如果是外文文献,此字段可忽略;如果是中文文献,此字段\emph{必须}填写|zh|。

{\BibTeX}需要根据此字段判别文献的语言,对于中文文献和外文文献采用不同的排版方式。

选择文献语言的基本原则是,若该文献本身是中文的(即使是翻译版),其对应项的语言就应该选|zh|,
其标题、作者等也应该尽量以中文填写。

\subsection{author}\label{subsec:author}

|author|字段表示文献的作者。

|author|字段可以有多个作者名。多个作者姓名之间\emph{必须}用英文词``and''隔开。根据
\cite{gbt7714-2005}的规定,{\BibTeX}会自动只使用前三个作者姓名进行排版,若还有多余作者,
会在作者姓名之后加``, 等''(对于中文文献)或``, et al''(对于外文文献)。

外文作者名建议按照\cite[157]{lamport1994latex}中的要求填写,{\BibTeX}系统会自动将其
转换为符合\std{GB/T 7714-2005}规范\cite{gbt7714-2005}的格式。具体可以参见
\ref{sec:names}。例如:
\begin{itemize}
\item |author = {Rogers, Jr., Hartley}|
\item |author = {Robert L. Constable and Stuart F. Allen}|
\end{itemize}

中文作者名采用姓在前名在后的方式填写,姓和名之间不需加空格。多个作者姓名之间也\emph{必须}用
英文词``and''隔开。例如:
\begin{itemize}
\item |author = {赵凯华}|
\item |author = {赵凯华 and 罗蔚茵 and 张三 and 李四 and 王五}|
\end{itemize}

中国作者姓名的汉语拼音的拼写执行\std{GB/T 16159—2012}的规定\cite{gbt16159-2012},
名字不能缩写。例如:``Zheng Guangmei''或``ZHENG Guangmei''。为了防止{\BibTeX}将
中国作者姓名的汉语拼音缩写,应该在该作者名两端用``{}''将其括起来。例如:
\begin{itemize}
\item |author = {{ZHAO Kaihua}}|
\item |author = {{ZHAO Kaihua} and {LUO Weiyin} and {ZHANG Sang}}|
\end{itemize}

对于中文文献,若其是外文文献的翻译版,应将其作者名写成中译名。

外国作者的中译名可以只著录其姓;同姓不同名的外国作者,其中译名不仅要著录其姓,还需著
录其名。外国作者的名可以用缩写字母,缩写名后可省略缩写点。翻译者的姓名应使用|translator|
字段指定。例如:
\begin{itemize}
\item |author = {马克思 and 恩格斯}|
\item |author = {昂温, G. and 昂温, P. S.}|
\end{itemize}

凡是对文献负责的机关团体名称通常根据著录信息源著录。用拉丁文书写的机关团体名称应由上至下分级著录 

如果文献的主要责任者是某个组织,例如公司或政府部门,其对应的文献项的|author|字段
应该填写该组织的名称。机关团体名称应由上至下分级著录。例如:``中国科学院数学研究所'',
``Stanford University, Department of Civil Engineering''。

为了防止{\BibLaTeX}将其识别为一个姓名,应该在组织名称两端用``{}''将其括起来。例如:
\begin{itemize}
\item |author = {{American Chemical Society}}|
\item |author = {{广西壮族自治区林业厅}}|
\end{itemize}

\subsection{editor}\label{subsec:editor}

|editor|字段表示文献的编辑者。其格式的具体要求与|author|字段类似。

|editor|字段和|author|字段的区别在于:|author|是所引用的文献的主要负责人,
|editor|则是所引用文献本身,或其所在的合集的编辑。例如:
\begin{itemize}
\item 如果所文献是一本书,但每个章节由不同的作者所写,而整本书由某个编辑者编辑。此时,
  \begin{itemize}
  \item 如果论文中引用的是全书,则应该将该文献作为|book|类型,其|editor|字段为编辑
    者,|author|字段不填。
  \item 如果论文中引用的是全书中由某个作者写的某个章节,则应该将文献作为
    |inbook|或|incollection|类型(取决于该章节是否有自己的标题,详见
    \ref{sec:bib-type}的解释),其|author|字段为该章节作者,|editor|字段为
    书籍的编辑者。
  \end{itemize}
\item 如果文献是会议论文集,此时:
  \begin{itemize}
  \item 如果论文引用的是论文集全集,则应该将文献作为|proceedings|类型,其|editor|字段
    为该论文集的编辑,|author|字段不填;
  \item 如果论文引用的是论文集中的某篇文章,则应该将文献作为|inproceedings|类型,
    其|author|字段为该文章作者,|editor|字段为该论文集的编辑者。
  \end{itemize}
\end{itemize}

关于|editor|和|author|字段的详细解释,请参见\ref{sec:bib-type}。

\subsection{translator}\label{subsec:translator}

|translator|字段表示文献的翻译者。其格式的具体要求与|author|字段类似。

例如:
\begin{itemize}
\item |translator = {张三}|
\item |translator = {张三 and 李四}|
\item |translator = {Bill Gates}|
\item |translator = {Bill Gates and Bill Clinton}|
\end{itemize}

\subsection{title}\label{subsec:title}

|title|字段表示文献的标题。如果该文献是析出文献(即某个专著中的某一章、某一节或某篇文章),
则|title|字段应该是被引用的析出文章或章节的标题;否则,|title|字段应该是被引用的文献的标题。

若被引用的文献的标题有副标题或其他标题,应将其与主标题用冒号隔开。

例如:
\begin{itemize}
\item |title = {Introduction to Algorithm}|
\item |title = {The Art of Computer Programming: Volume 3, Sorting and Searching}|,
\item |title = {马克思恩格斯全集: 第44卷}|
\end{itemize}

\subsection{booktitle}\label{subsec:booktitle}

|booktitle|字段表示析出文献所在的专著的标题。其要求与|title|字段类似。

\subsection{edition}\label{subsec:edition}

|edition|字段表示书籍的版本号。

该字段值有两种可取类型:
\begin{itemize}
\item 如果该字段的值是一个整数,则该整数表示书籍的版本号。{\BibTeX}会使用类似``3版''或
``3rd ed''这样的方式排版版本号;
\item 如果书籍是第一版,则无需此字段;
\item 如果该字段不是一个整数,则被当做一个字符串。{\BibTeX}会直接把该字符串排版在版本号的位置,
不做任何处理。
\item 如果是用英文表示的版本,通常应该以`` ed''结尾;如果是中文表示的版本,通常应该
以``xx版''的形式,序数前面无需加``第''字。
\end{itemize}

例如:
\begin{itemize}
\item 直接用整数表示版本号:|edition = {2}|
\item 以年代表示版本号:|edition = {1994 ed}|
\item 以年代表示版本号:|edition = {1994版}|
\item 其他版本号:|edition = {石印本}|
\end{itemize}

\subsection{address}\label{subsec:address}

\subsection{publisher}\label{subsec:publisher}

\subsection{year}\label{subsec:year}

|year|字段表示文献的出版年。

出版年采用公元纪年,并用阿拉伯数字著录。如有其他纪年形式时,将原有的纪年形式置于圆括号内。
例如:
\begin{itemize}
\item |year = {1947(民国三十六年)}|
\item |year = {1705(康熙 四十四年)}|
\end{itemize}

出版年无法确定时,可依次选用版权年、印刷年、估计的出版年。估计的出版年需置于方括号内。
例如:
\begin{itemize}
\item 版权年:|year = {c1998}|
\item 印刷年:|year = 1995印刷|
\item 估计的出版年:|year = {[1936]}|
\end{itemize}

\subsection{pages}\label{subsec:pages}

|pages|字段表示析出文献所在的页码或页码范围。页码用阿拉伯数字表示。页码范围用两个阿拉伯数字
分别表示起始页页码和结束页页码,两者之间用一个短横隔开。注意,页码字段不要包含诸如``p.'',
``page'',``页''之类的描述字符。例如:
\begin{itemize}
\item |pages = {128}|
\item |pages = {128--234}|
\end{itemize}

\subsection{citedate}\label{subsec:citedate}

|citedate|表示在线版文献的引用日期。此字段不是标准的{\BibTeX}字段,而是由{\njuthesis}
文档类所扩展的字段。

引用日期必须以阿拉伯数字表示年、月、日。“年”用四位阿拉伯数字表示;“月”用两位阿拉伯数字表示,
不足补零;“日”用两位阿拉伯数字表示,不足补零。三个部分之间用一个短横隔开。
例如:
\begin{itemize}
\item |citedate = {2013-08-30}|
\end{itemize}

如果文献项有|url|或|doi|字段,则也应有相应的|citedate|字段。

\subsection{url}\label{subsec:url}

|url|字段表示文献的在线URL地址。例如:
\begin{itemize}
\item |url = {http://www.di.ens.fr/users/longo/download.html}|
\end{itemize}

如果某文献项有|url|字段,则该文献被认为是在线文献。{\BibTeX}处理后,会在其文献类型后增
加``/OL'',表示其属于在线文献。

如果某文献项有|url|字段,则它也应该有对应的|citedate|字段。

\subsection{doi}\label{subsec:doi}

|doi|字段表示文献的``数字对象标识号(Digital Object Identifier)''。例如:
\begin{itemize}
\item |doi = {10.1007/s00223-003-0070-0}|
\end{itemize}

数字对象识别号是一套识别数字资源的机制,涵括的对象有视频、报告或书籍等等。它既有一套为资源命名的
机制,也有一套将识别号解析为具体地址的协议。

发展DOI的动机在于补充URI之不足,因为一方面URI指涉的URL经常变动,另一方面,URI表达的其实是资源
所在地(即网址),而非数字资源本身的信息。DOI能克服这两个问题。

一个DOI识别号经过解析后,可以连至一个或更多的数据。但识别号本身与解析后导向的数据并不相干,也可
能发生无法取得全部数据,只能得到相关出版品信息的情形。DOI的解析协议见诸RFC 3652,RFC 3651描
述命名机制,RFC 3650描述的则是其架构。DOI通过Handle系统解析识别号,但实际应用上大多是通过网
站解析.例如连进网址\url{http://dx.doi.org/10.1007/s00223-003-0070-0},就能看到对应识
别号\texttt{10.1007/s00223-003-0070-0}的论文信息或全文。

如果某文献项只有|doi|字段但没有|url|字段,该文献也被认为是在线文献。{\BibTeX}处理后,会在
其文献类型后增加``/OL'',表示其属于在线文献;同时也会自动根据DOI的解析网址前缀
\url{http://dx.doi.org/}合成其在线URL地址。

如果某文献项有|doi|字段,则它也应该有对应的|citedate|字段。

%%%%%%%%%%%%%%%%%%%%%%%%%%%%%%%%%%%%%%%%%%%%%%%%%%%%%%%%%%%%%%%%%%%%%%%%%%%%%%%

\section{姓名的格式}\label{sec:names}

\subsection{中国人的中文姓名}

\subsection{中国人的英文姓名}

\subsection{外国人的外文姓名}

\subsection{外国人名的中文译名}


