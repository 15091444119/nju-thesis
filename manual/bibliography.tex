\chapter{参考文献数据库}\label{chap:bib}

%%%%%%%%%%%%%%%%%%%%%%%%%%%%%%%%%%%%%%%%%%%%%%%%%%%%%%%%%%%%%%%%%%%%%%%%%%%%%%%
\section{简介}\label{sec:bib-intro}


%%%%%%%%%%%%%%%%%%%%%%%%%%%%%%%%%%%%%%%%%%%%%%%%%%%%%%%%%%%%%%%%%%%%%%%%%%%%%%%
\section{格式}\label{sec:bib-format}

%%%%%%%%%%%%%%%%%%%%%%%%%%%%%%%%%%%%%%%%%%%%%%%%%%%%%%%%%%%%%%%%%%%%%%%%%%%%%%%
\section{文献类型}\label{sec:bib-type}

\subsection{书籍}

书籍类文献所对应的{\BibTeX}文献项类型为|book|。

该类型文献的其必需字段包括:|author|、|title|、|publisher|、|year|。

其可选字段包括:|editor|、|translator|、|edition|、|volume|、|number|、|series|、
|address|、|url|、|doi|。

其他字段将不起作用。

\subsection{专著中的析出文献}

\subsection{期刊论文}

\subsection{会议论文}

\subsection{技术报告}

\subsection{学位论文}

\subsection{未发表论文}

\subsection{手册}

\subsection{标准}

\subsection{专利}

\subsection{网页}

%%%%%%%%%%%%%%%%%%%%%%%%%%%%%%%%%%%%%%%%%%%%%%%%%%%%%%%%%%%%%%%%%%%%%%%%%%%%%%%

\section{文献项字段}\label{sec:bib-field}

\subsection{language}

|language|字段用于说明该文献项对应的文献所使用的语言。

\begin{note}
如果是外文文献,此字段可忽略;如果是中文文献,此字段\emph{必须}填写|zh|。
\end{note}

{\BibTeX}需要根据此字段判别文献的语言,对于中文文献和外文文献采用不同的排版方式。

\subsection{author}

|author|字段表示文献的作者。

|author|字段可以有多个作者名。多个作者名之间\emph{必须}用英文词``and''隔开。根据
\cite{gbt7714-2005}的规定,{\BibTeX}会自动只使用前三个作者名进行排版,若还有多余作者,
会在作者名之后加``, 等''(对于中文文献)或``, et al''(对于外文文献)。

\beign{note}
即使文献作者名都是中文名,多个作者名之间也\emph{必须}用英文词``and''隔开。
\end{note}

外文作者名建议按照\cite[157]{lamport1994latex}中的要求填写,{\BibTeX}系统会自动将其
转换为符合\cite{gbt7714-2005}规范的格式。

中文作者名采用姓在前名在后的方式填写,姓和名之间不需加空格。

外文作者的中文翻译名必须按照\cite{gbt7714-2005}中的要求,要么只有其姓的中文译名,
要么可选择在其姓的中文译名后增加其名的缩写字母,例如:``昂温 P S'',缩写名后面
的``.''可以省略。

\begin{note}
对于中文文献,若其是外文文献的翻译版,应将其作者名写成中文译名。翻译者的姓名应使
用|translator|字段指定。
\end{note}

\subsection{editor}

|editor|字段表示文献的编辑者。其具体要求和|author|字段类似。

|editor|字段和|author|字段的区别在于:|author|是所引用的文献的主要负责人,
|editor|则是所引用文献本身,或其所在的合集的编辑。例如:
\begin{itemize}
\item 如果所文献是一本书,但每个章节由不同的作者所写,而整本书由某个编辑者编辑。此时,
  \begin{itemize}
  \item 如果论文中引用的是全书,则应该将该文献作为|book|类型,其|editor|字段为编辑
    者,|author|字段不填。
  \item 如果论文中引用的是全书中由某个作者写的某个章节,则应该将文献作为
    |inbook|或|incollection|类型(取决于该章节是否有自己的标题,详见
    \autoref{sec:bib-type}的解释),其|author|字段为该章节作者,|editor|字段为
    书籍的编辑者。
  \end{itemize}
\item 如果文献是会议论文集,此时:
  \begin{itemize}
  \item 如果论文引用的是论文集全集,则应该将文献作为|proceedings|类型,其|editor|字段
    为该论文集的编辑,|author|字段不填;
  \item 如果论文引用的是论文集中的某篇文章,则应该将文献作为|inproceedings|类型,
    其|author|字段为该文章作者,|editor|字段为该论文集的编辑者。
  \end{itemize}
\end{itemize}

关于|editor|和|author|字段的详细解释,请参见\autoref{sec:bib-type}

\subsection{edition}

|edition|字段表示书籍的版本号。

该字段值有两种可取类型:
\begin{itemize}
\item 如果该字段的值是一个整数,则该整数表示书籍的版本号。{\BibTeX}会使用类似``3版''或
``3rd ed''这样的方式排版版本号;
\item 如果书籍是第一版,则无需此字段;
\item 如果该字段不是一个整数,则被当做一个字符串。{\BibTeX}会直接把该字符串排版在版本号的位置,
不做任何处理。例如:``1994版'',``复刻版''等。
\end{itemize}

%%%%%%%%%%%%%%%%%%%%%%%%%%%%%%%%%%%%%%%%%%%%%%%%%%%%%%%%%%%%%%%%%%%%%%%%%%%%%%%

\section{姓名的格式}

\subsection{中国人的中文姓名}

\subsection{中国人的英文姓名}

\subsection{外国人的外文姓名}

\subsection{外国人名的中文译名}


